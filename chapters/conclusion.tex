\quad PCG-г ашигласнаар тоглоомын хөгжүүлэлтийн явцыг хөнгөвчлөх, нөөцийг үр ашигтай ашиглах бүрэн боломжтой. Гэхдээ төлөвлөгөө, үр дүнгээ тодорхой болгож ямар систем яаж хийхээ сайн мэдэхгүй бол систем нь хангалтгүй болж цаашлаад тоглоомын хөгжүүлэлтийн явцад саад болох эрсдэлтэй. Сайн PCG-н системийн боловсруулахад хэцүү. Мөн зарим тоглоомын төсөлд PCG-г тоглоомын явцад хэрэгжүүлэх хүсэлгүй байж болох юм. Гэсэн хэдий ч PCG-г тоглоомын явцад оруулж өгсөн амжилттай жишээнүүд олон байдаг. PCG-н системийг хэрэгжүүлэх аргад суралцан, туршилт хийснээр уг чиглэлээр илүү сайжрах боломжтой ба PCG-н системийг илүү оновчтой хэрэгжүүлж болон өөр олон аргатай хослуулан өвөрмөц сайн бүтээлийг бага өртгөөр гаргаж болно.

\section*{Талархал}
Судалгааны ажлын явцад мэдлэг болон судалгаа хийх боломжоор хангаж удирдсан Проф. Х. Оюундолгор багшдаа баярлалаа.