Процедурын дагуу контентыг үүсгэх (Procedural Content Generation буюу PCG) гэдэг нь хэрэглэгчийн хязгаарлагдмал болон шууд бус оролцоотойгоор алгоритм ашиглан контентыг үүсгэх явдал юм \cite{PCG}. Процедурын дагуу контентыг үүсгэх нь богино хугацаанд бага өртгөөр их хэмжээний контентыг үүсгэх боломжийг олгодог. Орчин үед тоглоомын контентыг процедурын дагуу үүсгэсэн олон жишээ бий. Гэхдээ том тоглоомуудын хувьд тоглоомын явцад нөлөөлөх байдал нь энгийн, сулавтар байгаа. Иймээс бид энэхүү өгүүллээр видео тоглоомын хувьд PCG-ийн талаар дэлгэрүүлэн судлан энэ аргыг видео тоглоомын салбарт яагаад өргөнөөр тоглоомын явцад оруулж өгөхгүй байгаа талаар шинжилж, нөөцийг хэмнэхэд өөр ямар боломжууд байгаа талаар дүгнэн хэлэлцэнэ.
Өгүүллийн дараагийн хэсгүүдэд dungeon-crawler төрлийн тоглоомын орчныг процедурын дагуу үүсгэх системийн алгоритмын сонголт, гүйцэтгэл, туршилт, эцэст нь туршилтаар олсон үр дүнд тулгуурлан дэвшүүлсэн асуултад хариулж, хэлэлцүүлэг, дүгнэлт хийнэ.
Энэхүү судалгааны ажилд dungeon-crawler төрлийн тоглоомын орчныг хэрхэн үүсгэхийг авч үзэх бөгөөд үүнд олон өрөө, тасалгаатай орчин багтана.

\section{Тоглоомын контент}
Тоглоомын контентод тоглоомын түүх, тоглоомын орчин, модель, моделийн материал зэрэг орж болох ба тоглоомыг хийхэд ашигласан engine, цаад систем зэрэг нь орохгүй. Өөрөөр хэлбэл тоглоомын контент нь тоглоомыг бүрэлдүүлж буй системүүдийн үр дүн, тоглогч тоглоомтой харилцах хэлбэр юм.

Тоглоомын контентыг компьютерын тусламжтайгаар үүсгэвэл артист, хөгжүүлэгч нарын цагийг ихээр хэмнэн нөөцийг үр ашигтай ашиглах боломжтой. Видео тоглоомыг үйлдвэрлэхэд үнэ өртөг жил ирэх тусам өссөөр байгаа ба уг салбар нь асар том юм. Хүн хүч болон цаг хугацаа нь өртөг ихтэй ба энэхүү өрсөлдөөн ихтэй салбарт нөөцийг үр ашигтай ашиглавал тухайн бүтээгдэхүүн нь амжилттай байх магадлал нь нэмэгдэнэ харин PCG-г ашигласнаар нөөцийг илүү үр дүнтэйгээр ашиглан илүү сайн, бүтээлч бүтээгдэхүүнийг үйлдвэрлэх боломжтой.

PCG-г орчин үед том тоглоомууд ихээр ашиглаж байгаа гэхдээ ашиглах хэлбэр нь энгийн, тоглоомын явцад нөлөөлөх хэлбэр нь сул байдаг. Үүнд орчин үеийн AAA тоглоомуудын модны моделийг процедурын аргаар үүсгэдэг гол дундын систем болох SpeedTree, процедурын дагуу үүссэн моделийн материалын жишээ зэрэг орж болно. Модны модель болон материал зэрэг нь тоглоомын контент мөн гэхдээ тоглоомын явцад нөлөөлөх нь бага.

\section{Тоглоомын явц}
Тоглоомын явц нь тоглогч тоглоомтой харилцах гол хэлбэр ба тухайн тоглоомоос шалтгаалаад энэ нь янз бүр байж болно. Тоглогч нь тоглоомтой олон хэлбэрээр харилцаж болно харин тэр болгон нь тоглоомын явцад орохгүй. Мэдээж бүх тоглоомд гол цэс эсвэл хэрэглэгчийн харилцах хэсэг байдаг харин тухайн тоглоом нь хэрэглэгчийн харилцах хэсгээрээ тоглогчоо татаж эсвэл энэ нь гол тоглогдох хэлбэр нь биш л бол үүнийг тоглоомын явцад нөлөөлж байна гэж үзэхэд хэцүү. Тоглоомын явцад харин Minecraft тоглоомын эрх чөлөөтэйгөөр блокуудыг удирдах орж болно. Тоглоом нь тоглоомын явцаар тоглогчийг уусган, өөрийг нь илэрхийлэх хэлбэрийг тодорхойлдог. Тоглоомд тоглоомын явц чухал нөлөөтэй.

Тоглоомын явцад нөлөөлөх контентуудыг харин процедурын дагуу үүсгэвэл тоглоомын явцыг процедурын дагуу үүсгэх боломжтой. Ингэснээр тоглоомын дахин тоглогдох чадвар сайжирч тоглолт болгон нь шинэ содон өөр байх боломжтой ба энэ нь бүтээгдэхүүний үнэ цэнд сайнаар нөлөөлнө.

\section{PCG-н шийдлийн шинж чанар}
PCG нь нэг талдаа компьютер болон алгоримтийн тусламжтайгаар хөгжүүлэгчийн оролцоог багасган контентыг үүсгэх үйл явц. Үүнд тухайн контентыг үүсгэх үйл явцын шийдэл хамаарагдана.

\subsection{PCG болон үүний шийдлийн хүсэмжид шинж чанарууд\cite{PCGinGames2016}:}
\begin{itemize}
	\item Хурд

	      Зарим тоглоомын контентыг ажиллах хугацааны явцад эвсэл тоглоомын хөгжүүлэлтийн явцад ашиглагдахаас шалтгаалаад үүсгэх хугацаа нь янз бүр байж болно. Тухайлбал тоглоомын орчин нь тоглогчийн хариу үйлдлээс хамаарч үүсгэгддэг бол шийдлийн хурд нь 2 милл секундийн дотор байна.
	\item Найдвартай байдал

	      Мөн тухайн контентын шаардлагаас хамаараад энэхүү шинж чанарын биелэгдэх байдал өөр байна. Тоглоомын орчны ердийн өвс эвсэл цэцгийг үүсгэхэд үүний зөв эсэх нь чухал биш харин тоглоомын орчны өрөөнүүдийн ерөнхий бүтцийг гаргаж байвал аливаа өрөө нь орох гарах багадаа хоёр хаалга байх зэрэг шаардлага байж болно.
	\item Удирдах чадвар

	      Энд тухайн шийдлийн үр дүн буюу контентын шинж чанарыг хэрэглэгч өөрчлөх эсвэл удирдах орно. Жишээлбэл өвсний урт өргөн, өнгө зэргийг хэрэглэгчид төвөг удахгүй өөрчлөх орж болно.
	\item Илэрхийлэлт болон олон талт байдал

	      Аливаа нэг сэдэвтэй олон төрлийн контентыг үүсгэх шаардлага тоглоомд тулгарч байдаг. Жишээлбэл тоглоомын орчны өрөөг үүсгэхэд олон талт байдлыг хэрэгжүүлж өгснөөр тухайн өрөөний дотор засал бусад өрөөнөөс өөр байх зэрэг орж болно.
	\item Бүтээлч байдал болон итгэж болох байдал

	      Процедурт аргаар үүсгэгдсэн контентод тоглогч итгэж болох байх шаардлагатай. Жишээлбэл тоглоомын орчны өрөө хэт том биш эсвэл том бол дотроо тулгуур баганатай байх зэрэг орж болно.
\end{itemize}

\section{PCG гэдэгт дараах зүйлсийг хамруулахгүй}
\begin{itemize}
	\item Санамсаргүй эсэх
	      \begin{itemize}
		      \item PCG-н олон талт байдлаас шалтгаалж хүмүүс PCG-г санамсаргүй гэж бодох тохиолдол байдаг. Гэхдээ энэ нь олон алгоритм болон дүрмийн боловсруулалт ба санамсаргүй биш юм.
	      \end{itemize}
	\item Автомат эсэх
	      \begin{itemize}
		      \item PCG-д хэрэглэгчийн хариу үйлдлээс шалтгаалж боловсруулалт хийдэг шийдлүүд байдаг ба энэ нь автомат биш юм.
	      \end{itemize}
\end{itemize}

