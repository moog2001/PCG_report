\begin{abstract}
	\textbf{Видео тоглоом нь өндөр зардлаар бүтдэг, өрсөлдөөн ихтэй салбар юм. Учир нь тоглоомын контент нь тоглоомын түүх, тоглоомын орчин, дүр болон орчны модель, моделийн материал зэрэг олон хүчин зүйлсээс хамаардаг. Тоглоомын контентыг компьютерын тусламжтай үүсгэх нь хүн хүч, цаг хугацааг хэмнэх олон давуу талтай. Тиймээс энэхүү судалгааны ажлаар тоглоомын контентыг процедураар үүсгэх арга техникийг dungeon-crawler төрлийн тоглоомын орчныг үүсгэх жишээн дээр тайлбарлан харуулна.}

	\quad \quad \textbf{Түлхүүр үгс:} \textbf{\textit{Процедурын дагуу контентыг үүсгэх, тоглоомын контент, тоглоомын явц}}

	\quad \quad \textbf{Зорилго:}
	PCG-г видео тоглоомын хувьд шинжилгээ судалгаа хийн яагаад орчин үед үүнийг илүү ашиглахгүй байна бэ? мөн тоглоомын явцад нөлөөлөх боломжтой өөр содон арга хэлбэрийн талаар хэлэлцүүлэг эхлүүлэх.

	\quad \quad \textbf{Зорилт:}
	\begin{itemize}
		\item Unity дээр C\# хэл дээр олон давхар процедурын алгоримт нэгтгэн dungeon-crawler тоглоомын орчныг үүсгэх систем боловсруулах.
		      \begin{itemize}
			      \item Алгоритмын судалгаа
			      \item Харилцах хэсгийг боловсруулах
			      \item Системийг боловсруулах
			      \item Туршилт хийх
			      \item Сайжруулах
		      \end{itemize}
		\item Үр дүнгийн шинжилгээ
		      \begin{itemize}
			      \item Үр дүнгийн танилцуулга, шинжилгээ
			      \item PCG төрлийн контентыг тоглоомын явцад нөлөөлөх байдлаар ашиглах хэлбэрийн хэлэлцүүлэг
		      \end{itemize}
	\end{itemize}
\end{abstract}
