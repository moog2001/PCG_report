%----------------------------------------------------------------------------------------
%   Доорх хэсгийг өөрчлөх шаардлагагүй
%----------------------------------------------------------------------------------------
%!TEX TS-program = xelatex
%!TEX encoding = UTF-8 Unicode
\documentclass[12pt,A4]{report}

\usepackage{fontspec,xltxtra,xunicode}
\setmainfont[Ligatures=TeX]{Times New Roman}
\setsansfont{Arial}

% \usepackage[utf8x]{inputenc}
% \usepackage[mongolian]{babel}
%\usepackage{natbib}
\usepackage{geometry}
%\usepackage{fancyheadings} fancyheadings is obsolete: replaced by fancyhdr. JL
\usepackage{fancyhdr}
\usepackage{float}
\usepackage{afterpage}
\usepackage{graphicx}
\usepackage{amsmath,amssymb,amsbsy}
\usepackage{dcolumn,array}
\usepackage{tocloft}
\usepackage{dics}
\usepackage{nomencl}
\usepackage{upgreek}
\newcommand{\argmin}{\arg\!\min}
\usepackage{mathtools}
\usepackage[hidelinks]{hyperref}
\usepackage{bookmark}

\usepackage{algorithm}
\usepackage{algpseudocode}

\usepackage{listings}
\DeclarePairedDelimiter\abs{\lvert}{\rvert}%
\makeatletter
\usepackage{caption}
\captionsetup[table]{belowskip=0.5pt}
\usepackage{subfiles}

\usepackage{listings}
\renewcommand{\lstlistingname}{Код}
\renewcommand{\lstlistlistingname}{\lstlistingname ын жагсаалт}

% custom packages
\usepackage{subcaption}
\usepackage{biblatex}
\addbibresource{References.bib}
\usepackage{color}
\definecolor{codegreen}{rgb}{0,0.6,0}
\definecolor{codegray}{rgb}{0.5,0.5,0.5}
\definecolor{codepurple}{rgb}{0.58,0,0.82}
\definecolor{backcolour}{rgb}{0.99,0.99,0.99}
\definecolor{darkgray}{rgb}{0.99,0.103,0.105}
\definecolor{purple}{rgb}{0.128,0,0.128}
 
\lstdefinestyle{mystyle}{
    basicstyle=\ttfamily\small,
    backgroundcolor=\color{backcolour},   
    commentstyle=\color{codegreen},
    keywordstyle=\color{magenta},
    numberstyle=\tiny\color{codegray},
    stringstyle=\color{codepurple},
    %basicstyle=\footnotesize,
    breakatwhitespace=false,         
    breaklines=true,                 
    captionpos=b,                    
    keepspaces=false,                 
    numbers=left,                    
    numbersep=10pt,                  
    showspaces=false,                
    showstringspaces=true,
    showtabs=false,                  
    tabsize=2
}
 
 %define Javascript language
\lstdefinelanguage{JavaScript}{
keywords={typeof, new, true, false, catch, function, return, null, catch, switch, var, if, in, while, do, else, case, break},
keywordstyle=\color{blue}\bfseries,
ndkeywords={class, export, boolean, throw, implements, import, this},
ndkeywordstyle=\color{darkgray}\bfseries,
identifierstyle=\color{black},
sensitive=false,
comment=[l]{//},
morecomment=[s]{/*}{*/},
commentstyle=\color{purple}\ttfamily,
stringstyle=\color{red}\ttfamily,
morestring=[b]',
morestring=[b]"
}
 
\lstset{
language=JavaScript,
extendedchars=true,
basicstyle=\footnotesize\ttfamily,
showstringspaces=false,
showspaces=false,
numbers=left,
numberstyle=\footnotesize,
numbersep=9pt,
tabsize=2,
breaklines=true,
showtabs=false,
captionpos=b
}
 
\lstset{style=mystyle, label=DescriptiveLabel} 

%defining c#

% \definecolor{cloudwhite}{rgb}{0.9412, 0.9608, 0.8471}

% \lstset{
% language=csh,
% basicstyle=\footnotesize\ttfamily,
% numbers=left,
% numberstyle=\tiny,
% numbersep=5pt,
% tabsize=2,
% extendedchars=true,
% breaklines=true,
% frame=b,
% stringstyle=\color{blue}\ttfamily,
% showspaces=false,
% showtabs=false,
% xleftmargin=17pt,
% framexleftmargin=17pt,
% framexrightmargin=5pt,
% framexbottommargin=4pt,
% commentstyle=\color{green},
% morecomment=[l]{//}, %use comment-line-style!
% morecomment=[s]{/*}{*/}, %for multiline comments
% showstringspaces=false,
% morekeywords={ abstract, event, new, struct,
% as, explicit, null, switch,
% base, extern, object, this,
% bool, false, operator, throw,
% break, finally, out, true,
% byte, fixed, override, try,
% case, float, params, typeof,
% catch, for, private, uint,
% char, foreach, protected, ulong,
% checked, goto, public, unchecked,
% class, if, readonly, unsafe,
% const, implicit, ref, ushort,
% continue, in, return, using,
% decimal, int, sbyte, virtual,
% default, interface, sealed, volatile,
% delegate, internal, short, void,
% do, is, sizeof, while,
% double, lock, stackalloc,
% else, long, static,
% enum, namespace, string},
% keywordstyle=\color{cyan},
% identifierstyle=\color{red},
% backgroundcolor=\color{cloudwhite},
% }



\let\oldabs\abs
\def\abs{\@ifstar{\oldabs}{\oldabs*}}
\makenomenclature
\begin{document}


%----------------------------------------------------------------------------------------
%   Өөрийн мэдээллээ оруулах хэсэг
%----------------------------------------------------------------------------------------

% Дипломийн ажлын сэдэв
\title{Процедурын дагуу видео тоглоомын орчныг үүсгэх}
% Дипломын ажлын англи нэр
\titleEng{Procedurally Generating Video Game Enviroment}
% Өөрийн овог нэрийг бүтнээр нь бичнэ
\author{Соёл-Эрдэнэ Мөнх-Очир}
% Өөрийн овгийн эхний үсэг нэрээ бичнэ
\authorShort{С. Мөнх-Очир}
% Удирдагчийн зэрэг цол овгийн эхний үсэг нэр
\supervisor{Проф. Х. Оюундолгор}
% Хамтарсан удирдагчийн зэрэг цол овгийн эхний үсэг нэр

% СиСи дугаар 
\sisiId{19B1NUM0574}
% Их сургуулийн нэр
\university{МОНГОЛ УЛСЫН ИХ СУРГУУЛЬ}
% Бүрэлдэхүүн сургуулийн нэр
\faculty{ХЭРЭГЛЭЭНИЙ ШИНЖЛЭХ УХААН, ИНЖЕНЕРЧЛЭЛИЙН СУРГУУЛЬ}
% Тэнхимийн нэр
\department{МЭДЭЭЛЭЛ, КОМПЬЮТЕРИЙН УХААНЫ ТЭНХИМ}
% Зэргийн нэр
\degreeName{Бакалаврын судалгааны ажил}
% Суралцаж буй хөтөлбөрийн нэр
\programeName{Програм Хангамж(D061302)}
% Хэвлэгдсэн газар
\cityName{Улаанбаатар}
% Хэвлэгдсэн огноо
\gradyear{2023 оны 3 сар}


%--------------------------------------------------------`--------------------------------
%   Доорх хэсгийг өөрчлөх шаардлагагүй
%----------------------------------------------------------------------------------------
\include{important/main-pre}

\begin{abstract}
	\textbf{Видео тоглоом нь өндөр зардлаар бүтдэг, өрсөлдөөн ихтэй салбар юм. Учир нь тоглоомын контент нь тоглоомын түүх, тоглоомын орчин, дүр болон орчны модель, моделийн материал зэрэг олон хүчин зүйлсээс хамаардаг. Тоглоомын контентыг компьютерын тусламжтай үүсгэх нь хүн хүч, цаг хугацааг хэмнэх олон давуу талтай. Тиймээс энэхүү судалгааны ажлаар тоглоомын контентыг процедураар үүсгэх арга техникийг dungeon-crawler төрлийн тоглоомын орчныг үүсгэх жишээн дээр тайлбарлан харуулна.}

	\quad \quad \textbf{Түлхүүр үгс:} \textbf{\textit{Процедурын дагуу контентыг үүсгэх, тоглоомын контент, тоглоомын явц}}

	\quad \quad \textbf{Зорилго:}
	PCG-г видео тоглоомын хувьд шинжилгээ судалгаа хийн яагаад орчин үед үүнийг илүү ашиглахгүй байна бэ? мөн тоглоомын явцад нөлөөлөх боломжтой өөр содон арга хэлбэрийн талаар хэлэлцүүлэг эхлүүлэх.

	\quad \quad \textbf{Зорилт:}
	\begin{itemize}
		\item Unity дээр C\# хэл дээр олон давхар процедурын алгоримт нэгтгэн dungeon-crawler тоглоомын орчныг үүсгэх систем боловсруулах.
		      \begin{itemize}
			      \item Алгоритмын судалгаа
			      \item Харилцах хэсгийг боловсруулах
			      \item Системийг боловсруулах
			      \item Туршилт хийх
			      \item Сайжруулах
		      \end{itemize}
		\item Үр дүнгийн шинжилгээ
		      \begin{itemize}
			      \item Үр дүнгийн танилцуулга, шинжилгээ
			      \item PCG төрлийн контентыг тоглоомын явцад нөлөөлөх байдлаар ашиглах хэлбэрийн хэлэлцүүлэг
		      \end{itemize}
	\end{itemize}
\end{abstract}


\addcontentsline{toc}{part}{БҮЛГҮҮД}

\chapter{Танилцуулга}
\subfile{chapters/introduction}

\chapter{Алгоритмын судалгаа}
\subfile{chapters/research}

\chapter{Ашиглах технологи}
\subfile{chapters/technologies}

\chapter{Хэрэгжүүлэх систем}
\subfile{chapters/system}

\chapter{Үр дүн, Хэлэлцүүлэг}
\subfile{chapters/results}

\chapter{Дүгнэлт}
\subfile{chapters/conclusion.tex}

%----------------------------------------------------------------------------------------
%   Дипломын номзүй, хавсралтын хэсэг эндээс эхэлнэ
%----------------------------------------------------------------------------------------

\singlespace
\addcontentsline{toc}{part}{НОМ ЗҮЙ}
\printbibliography




%----------------------------------------------------------------------------------------
%   Хавсралтууд эндээс эхэлнэ
%----------------------------------------------------------------------------------------
\appendix
\addcontentsline{toc}{part}{ХАВСРАЛТ}

\chapter{Кодын хэрэгжүүлэлт}
\section{Системийн хэсэг}
\subsection{BinarySpaceTree}
\begin{lstlisting}[language={[Sharp]C}, frame=single, caption=BinarySpaceTree хэрэгжүүлэлт]
using System.Collections.Generic;
using System.Data;
using System.Linq;
using System.Text;
using CodeMonkey.Utils;
using Unity.VisualScripting;
using UnityEngine;
using UnityEngine.Rendering.UI;
using Random = UnityEngine.Random;


namespace DefaultNamespace
{
    public class BinarySpaceTree
    {
        public static int VERTICAL_CUT_INDEX = 0;
        public static int HORIZONTAL_CUT_INDEX = 1;

        private Rectangle _nodeRectangle;
        private BinarySpaceTree _leftChild;
        private BinarySpaceTree _rightChild;
        private BinarySpaceTree _parentCell;
        private bool _isLeaf = true;
        private int _level = 0;
        private ProceduralGenerationCellBundle _bundle;
        public int Key;

        public Rectangle NodeRectangle
        {
            get => _nodeRectangle;
            set => _nodeRectangle = value;
        }


        public bool IsLeaf => _isLeaf;

        public int Level => _level;

        public BinarySpaceTree RightChild
        {
            get => _rightChild;
            set
            {
                if (value == null)
                    return;
                _rightChild = value;
                _isLeaf = false;
            }
        }

        public BinarySpaceTree LeftChild
        {
            get => _leftChild;
            set
            {
                if (value == null)
                    return;
                _leftChild = value;
                _isLeaf = false;
            }
        }

        public BinarySpaceTree ParentCell
        {
            get => _parentCell;
            set
            {
                if (value == null)
                {
                    _parentCell = null;
                    _level = 0;
                }
                else
                {
                    _parentCell = value;
                    _level = _parentCell._level + 1;
                }
            }
        }


        public BinarySpaceTree(Rectangle nodeRectangle, ProceduralGenerationCellBundle bundle)
        {
            NodeRectangle = nodeRectangle;
            _bundle = bundle;
            ParentCell = null;
            Key = 1;
            Divide();
        }

        private BinarySpaceTree(Rectangle nodeRectangle, BinarySpaceTree parentCell,
            ProceduralGenerationCellBundle bundle, int key)
        {
            NodeRectangle = nodeRectangle;
            ParentCell = parentCell;
            _bundle = bundle;
            Key = key;
            Divide();
        }

        public void Divide()
        {
            if (CanMutateToBeBig()) return;
            int direction = GetDirectionToCut();
            if (direction == -1) return;
            CreateChildNodes(direction);
        }

        private void CreateChildNodes(int direction)
        {
            if (direction == VERTICAL_CUT_INDEX)
            {
                CreateChildNodesVertically();
            }
            else
            {
                CreateChildNodesHorizontally();
            }
        }

        private void CreateChildNodesHorizontally()
        {
            var divisionAmount = Random.Range(_bundle.MIN_ROOM_SIZE, NodeRectangle.Height + 1 - _bundle.MIN_ROOM_SIZE);
            divisionAmount = Math.Max(_bundle.MIN_ROOM_SIZE, divisionAmount);

            var leftChildCoords2D = new Coord2D(NodeRectangle.StartingCoord2D.X, NodeRectangle.StartingCoord2D.Z);
            var leftChildRectangle = new Rectangle(leftChildCoords2D, NodeRectangle.Width, divisionAmount);
            LeftChild = new BinarySpaceTree(leftChildRectangle, this, _bundle, Key * 2);

            var rightChildCoords2D = new Coord2D(NodeRectangle.StartingCoord2D.X,
                NodeRectangle.StartingCoord2D.Z + divisionAmount);
            var rightChildRectangle = new Rectangle(rightChildCoords2D, NodeRectangle.Width,
                NodeRectangle.Height - divisionAmount);
            RightChild = new BinarySpaceTree(rightChildRectangle, this, _bundle, Key * 2 + 1);
        }

        private void CreateChildNodesVertically()
        {
            var divisionAmount = Random.Range(_bundle.MIN_ROOM_SIZE, NodeRectangle.Width + 1 - _bundle.MIN_ROOM_SIZE);
            divisionAmount = Math.Max(_bundle.MIN_ROOM_SIZE, divisionAmount);

            var leftChildCoords2D = new Coord2D(NodeRectangle.StartingCoord2D.X, NodeRectangle.StartingCoord2D.Z);
            var leftChildRectangle = new Rectangle(leftChildCoords2D, divisionAmount, NodeRectangle.Height);
            LeftChild = new BinarySpaceTree(leftChildRectangle, this, _bundle, Key * 2);

            var rightChildCoords2D = new Coord2D(NodeRectangle.StartingCoord2D.X + divisionAmount,
                NodeRectangle.StartingCoord2D.Z);
            var rightChildRectangle = new Rectangle(rightChildCoords2D, NodeRectangle.Width - divisionAmount,
                NodeRectangle.Height);
            RightChild = new BinarySpaceTree(rightChildRectangle, this, _bundle, Key * 2 + 1);
        }

        private int GetDirectionToCut()
        {
            int direction;
            if (NodeRectangle.Width >= _bundle.MIN_ROOM_SIZE * 2 && NodeRectangle.Height < _bundle.MIN_ROOM_SIZE * 2)
                direction = VERTICAL_CUT_INDEX;
            else if (NodeRectangle.Width < _bundle.MIN_ROOM_SIZE * 2 &&
                     NodeRectangle.Height >= _bundle.MIN_ROOM_SIZE * 2)
                direction = HORIZONTAL_CUT_INDEX;
            else if (NodeRectangle.Width >= _bundle.MIN_ROOM_SIZE * 2 &&
                     NodeRectangle.Height >= _bundle.MIN_ROOM_SIZE * 2)
                direction = Random.Range(0, 2);
            else
                return -1;
            return direction;
        }

        private bool CanMutateToBeBig()
        {
            if (_level == 0) return false;
            if (NodeRectangle.Width < _bundle.MAX_ROOM_SIZE && NodeRectangle.Height < _bundle.MAX_ROOM_SIZE)
            {
                // multiplying because the generator has 0.5 chance multiplier
                // max bundle leaf node input is 50, so max is 50%
                if (RandomUtils.Chance(_bundle.LEAF_NODE_CHANCE * 2))
                    return true;
            }

            return false;
        }

        public void AssignNodesAtLevelFromRootNode(int level, List<BinarySpaceTree> nodesAtLevel)
        {
            if (Key != 1)
                throw new InvalidOperationException("This is not the root node");
            AssignNodesAtLevelForNode(level, nodesAtLevel);
        }

        private void AssignNodesAtLevelForNode(int level, List<BinarySpaceTree> nodesAtLevel)
        {
            if (Level == level)
            {
                nodesAtLevel.Add(this);
            }

            if (LeftChild != null)
                LeftChild.AssignNodesAtLevelForNode(level, nodesAtLevel);
            if (RightChild != null)
                RightChild.AssignNodesAtLevelForNode(level, nodesAtLevel);
        }


        public void AssignLeafNodesFromRootNode(List<BinarySpaceTree> leafNodes)
        {
            if (Key != 1)
                throw new InvalidOperationException("This is not the root node");
            AssignLeafNodes(leafNodes);
        }

        private void AssignLeafNodes(List<BinarySpaceTree> leafNodes)
        {
            if (IsLeaf)
            {
                leafNodes.Add(this);
                return;
            }

            LeftChild.AssignLeafNodes(leafNodes);
            RightChild.AssignLeafNodes(leafNodes);
        }

        public BinarySpaceTree GetSibling()
        {
            if (ParentCell == null)
                throw new InvalidOperationException("This node doesn't have a parent");
            if (Key != ParentCell.LeftChild.Key)
                return ParentCell.LeftChild;

            return ParentCell.RightChild;
        }

        public bool IsRightChild()
        {
            if (ParentCell == null)
                throw new InvalidOperationException("This node doesn't have a parent");
            if (ParentCell.RightChild.Key == Key)
            {
                return true;
            }

            return false;
        }

        public static int GetLevelDepth(BinarySpaceTree binarySpaceTree, int initialLevel = 1)

        {
            if (binarySpaceTree.Key != 1)
                throw new InvalidOperationException("This is not the root node");

            List<BinarySpaceTree> nodesAtTheBottom = new List<BinarySpaceTree>();
            binarySpaceTree.AssignLeafNodes(nodesAtTheBottom);

            nodesAtTheBottom.ForEach(e =>
            {
                if (e.Level > initialLevel)
                {
                    initialLevel = e.Level;
                }
            });
            return initialLevel;
        }


        public BinarySpaceTree GetNodesAtIndex(int index)
        {
            if (this.Key != 1)
                throw new InvalidOperationException("This is not the root node");
            return GetNodeAtIndexHelper(this, index);
        }

        // in-order traversal
        private static BinarySpaceTree GetNodeAtIndexHelper(BinarySpaceTree node, int index)
        {
            if (node == null)
                return null;

            if (node.Key == index)
                return node;
            BinarySpaceTree leftNode = GetNodeAtIndexHelper(node.LeftChild, index);

            if (leftNode != null)
            {
                if (leftNode.Key == index)
                    return leftNode;
            }

            return GetNodeAtIndexHelper(node.RightChild, index);
        }

        public static void InorderTraversalDebugLog(BinarySpaceTree root)
        {
            if (root != null)
            {
                InorderTraversalDebugLog(root.LeftChild);
                Debug.Log(root.Key + " ");
                InorderTraversalDebugLog(root.RightChild);
            }
        }

        public Coord2D GetStartingCoords()
        {
            return NodeRectangle.StartingCoord2D;
        }
    }
}
\end{lstlisting}
\chapter{Ажлын төлөвлөгөө}
\begin{figure}[b]
	\centering
	\includegraphics[angle=90,height=\textheight]{./images/plan.pdf}
	\caption{Ажлын төлөвлөгөө}
	\label{fig:WorkPlan}
\end{figure}

\end{document}
